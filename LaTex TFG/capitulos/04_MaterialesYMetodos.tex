\chapter{Materiales y métodos}
\thispagestyle{empty}

1. Van Dijk, T.,  De Croon, G. (2019). How do neural networks see depth in single images? In IEEE/CVF international conference on computer vision (pp. 2183–2191).

\section{Materiales}

\subsection{Modelos 3D}
En este apartado se describe cómo se obtuvieron los conjuntos de datos 3D y las decisiones tomadas para elegir unos dataset frente a otros. Además, se explicará el subconjunto de casos elegidos para generar las fotografías faciales 2D.

\subsubsection{Modelos faciales}
Aquí sobre los modelos HeadSpace, H3DS-net, ...

\subsubsection{Modelos de cuerpo entero}
Aquí sobre HuMMan, People Snapshot, ...

\subsection{Procesamiento del conjunto de datos}
En este apartado explicaremos la necesidad de alinear los modelos 3D para tener el origen a la altura de los ojos y así poder tener una referencia a la hora de generar imágenes 2D a distintas distancias. Se expondrán los métodos llevados a cabo para dicha finalidad (tanto alinear como generar las imágenes). Además, se explicará cómo se han cambiado los fondos e iluminación de las imágenes para hacerlas más realistas.

\section{Métodos}

\subsubsection{Alineamiento de modelos 3D}

\subsubsection{Generación de fotografías faciales a partir de modelos 3D}

\subsubsection{Mejoras en fondo e iluminación de imágenes}

En los siguientes apartados se describen las arquitecturas de deep learning que vamos a utilizar para realizar los experimentos.

\subsection{FacialSCDnet+}

