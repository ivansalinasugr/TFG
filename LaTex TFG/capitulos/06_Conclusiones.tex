\chapter{Conclusiones y trabajos futuros}

\thispagestyle{empty}

La distancia entre la cámara y el sujeto en fotografías faciales es un tema relevante en diversas disciplinas, debido a los cambios en la apariencia de los sujetos fruto del impacto de la distorsión de perspectiva. Para facilitar el análisis de estas imágenes, en el presente TFG, nos enfocamos en desarrollar un método basado en técnicas \textit{deep learning} para la estimación de la distancia cámara-sujeto (SCD) de manera robusta en escenarios reales. Para ello, se cumplieron satisfactoriamente una serie de objetivos fundamentales. Por un lado, se realizó un análisis del estado del arte y del método de referencia FacialSCDnet \cite{14}; se examinaron diferentes bases de datos existentes de modelos faciales y humanos en 3D; y se diseñó un protocolo de estandarización para generar imágenes sintéticas fotorrealistas a partir de estos modelos. Además, se diseñó e implementó el método propuesto junto con un sistema de gestión de experimentos basado en la metodología MLOps; se llevó a cabo un estudio comparativo de la nueva aproximación propuesta; y, finalmente, se exploraron nuevas arquitecturas en busca de un mejor rendimiento respecto al método original.

La propuesta desarrollada en este trabajo, FacialSCDnet+, aborda las limitaciones identificadas sobre el método principal del estado del arte \cite{14}. Esta propuesta reduce aproximadamente a la mitad el error en la estimación, además de reducir los sesgos producidos en FacialSCDnet por el preprocesamiento original de las imágenes. Para ello, se diseñó un protocolo novedoso para crear un conjunto de imágenes sintéticas generadas a partir de modelos 3D, que sirvieron para entrenar dos modelos de aprendizaje basados en las arquitecturas VGG-16 y ResNet-50. Ambos modelos se adaptaron a la tarea de regresión de la distancia, mostrando buenos resultados en cuanto a predicciones tanto en conjuntos de test sintéticos como reales. En concreto, el modelo VGG-16 superó el rendimiento del modelo ResNet-50 e incluso del modelo FacialSCDnet, logrando un error medio de distorsión menor al 1\% usado como referencia, tanto en el conjunto sintético como en el real. Este hecho se observa en todo el rango de distancias, sin embargo, el impacto de la métrica de distorsión es evidente a distancias mayores, ya que sigue presente cierta dispersión en las estimaciones. Este comportamiento es esperable, ya que a medida que avanza la distancia, el impacto de la distorsión de perspectiva se reduce exponencialmente, y es precisamente esta distorsión la que guía el aprendizaje del modelo. Así, podemos concluir que todas las propuestas realizadas en este trabajo han contribuido a superar el rendimiento del estado del arte para la estimación de la SCD con técnicas de \textit{deep learning}. Además, gracias al sistema implementado y los experimentos realizados se han detectado una serie de fallos concretos del planteamiento original FacialSCDnet, que han permitido identificar algunos factores que deben tenerse en cuenta a la hora de trabajar con datos y modelos de aprendizaje automático, con el fin de evitar posibles sesgos en el proceso de entrenamiento. Este conocimiento es muy valioso dentro del campo de la investigación y el desarrollo de modelos de inteligencia artificial.

En general, para el desarrollo de este TFG, se utilizaron conocimientos tanto de la asignatura de Aprendizaje Automático como de Visión por Computador. Además, se adquirieron nuevos conocimientos sobre modelado 3D con Blender, y el uso de herramientas desconocidas para el autor como Keras.

De cara a los trabajos futuros, se presentan varios desafíos y oportunidades de mejora. Uno de los principales desafíos es la predicción robusta de distancias más lejanas, donde las actuales arquitecturas pueden no ser suficientemente precisas, para ello se explorarían métricas alternativas que permitan tener en cuenta tanto la distancia métrica como el error de distorsión. Además, sería interesante explorar el impacto del uso de distintas focales en las predicciones, así como un diseño alternativo que integre el valor de la distancia focal para evitar la necesidad de entrenar varios modelos diferentes. Este es un factor importante a tener en cuenta, ya que las variaciones en la distancia focal pueden influir significativamente en la exactitud de la estimación de la SCD. Por último, se propone ampliar el uso de modelos 3D de cuerpo completo para aumentar la variablidad de poses de forma que el conjunto de datos sea más diverso y realista aún. Esto podría proporcionar una mejor representación de las variaciones anatómicas y mejorar la robustez del modelo ante diferentes condiciones y tipos de sujetos.
