\chapter{Conclusiones y trabajos futuros}

\thispagestyle{empty}

La distancia entre la cámara y el sujeto en fotografías faciales es un tema relevante en diversas disciplinas, debido a su impacto en la distorsión de los sujetos en las imágenes. En el TFG desarrollado, nos centramos en mejorar la estimación precisa de la distancia cámara-sujeto (SCD), utilizando técnicas de \textit{deep learning}. Para ello, se cumplieron satisfactoriamente una serie de objetivos fundamentales: se realizó un análisis del estado del arte y del método de referencia FacialSCDnet; se examinaron las bases de datos existentes de modelos faciales y humanos en 3D; se diseñó un protocolo de estandarización para generar imágenes sintéticas fotorrealistas a partir de estos modelos; se llevó a cabo un estudio comparativo de la nueva aproximación propuesta; y se exploraron nuevas arquitecturas en busca de un mejor rendimiento respecto al método original.

El método FacialSCDnet+ desarrollado en este trabajo utilizó un conjunto de imágenes sintéticas generadas a partir de modelos 3D, que sirvieron para entrenar dos modelos de aprendizaje: VGG-16 y ResNet-50. Ambos modelos se adaptaron a la tarea de regresión de la distancia, mostrando buenos resultados en cuanto a predicciones tanto en conjuntos de test sintéticos como reales. El modelo VGG-16 superó el rendimiento del modelo ResNet-50 e incluso del modelo FacialSCDnet. Este hecho se observa tanto a distancias cortas como largas. Sin embargo, aunque la distorsión a distancias lejanas es mucho menor, las predicciones podrían mejorar en cuanto a exactitud. A pesar de esto, el modelo VGG-16 de FacialSCDnet+ ha logrado superar el rendimiento del estado del arte para la estimación de la SCD con técnicas de \textit{deep learning}.

En general, para el desarrollo de este TFG, se utilizaron conocimientos tanto de la asignatura de Aprendizaje Automático como de la de Visión por Computador. Además, se adquirieron nuevos conocimientos sobre modelado 3D con Blender, y el uso de herramientas desconocidas para el autor como Pytorch o Keras.

De cara a los trabajos futuros, se presentan varios desafíos y oportunidades de mejora. Uno de los principales desafíos es la predicción precisa de distancias más lejanas, donde las actuales arquitecturas pueden no ser suficientemente precisas. Además, sería interesante explorar el impacto del uso de distintas focales en las predicciones, ya que las variaciones en la longitud focal pueden influir significativamente en la exactitud de la estimación de la SCD. Por último, se propone únicamente utilizar modelos 3D de cuerpo completo y de gran calidad para generar un conjunto de datos más diverso y realista. Esto podría proporcionar una mejor representación de las variaciones anatómicas y mejorar la robustez del modelo ante diferentes condiciones y tipos de sujetos.
