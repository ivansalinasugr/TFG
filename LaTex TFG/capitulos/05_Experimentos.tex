\chapter{Experimentos}
\thispagestyle{empty}

\section{Detalles técnicos de la implementación}

\subsection{Entorno de desarrollo}
En este apartado describimos el entorno que vamos a usar para entrenar el modelo.

\subsection{Obtención del conjunto de datos}
En este apartado se explica cómo se obtienen las imágenes que posteriormente se van a usar para entrenar el modelo. Este apartado no explica cómo se generan (eso sería en el 4.2.1) sino cómo se organizan (según longitud focal y distancia) ya que tenemos que entrenar 4 modelos.

\subsection{Entrenamiento del modelo}
En este apartado se describe qué hay que hacer para entrenar el modelo (ejecutar ciertos archivos) y en qué sistemas se va a entrenar (GPU de la UGR mediante SSH, etc)


\section{Experimentos}

\subsection{Protocolo de validación experimental}
Este apartado describimos el esquema de división del dataset en entrenamiento, validación y test.

\subsection{Métricas}
Este apartado explica las métricas que vamos a utilizar para medir el rendimiento de los modelos.

\subsection{Experimentos con VGG16}
En este apartado realizamos el entrenamiento de nuestros modelos con el nuevo dataset. Comparamos con los resultados de FacialSCDnet para ver si nuestro nuevo dataset mejora de algún modo el entrenamiento.

Además, en otro apartado se podrían utilizar otras arquitecturas distintas a VGG16 para comparar con esta.