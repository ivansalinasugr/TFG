\chapter{Fundamentos teóricos}
\thispagestyle{empty}

? Distancia cámara sujeto

? Distorsión de perspectiva

? Reconocimiento facial

? Transferencia de aprendizaje

\section{Aprendizaje automático}
El aprendizaje automático (machine learning, ML) \cite{16,17} es una rama de la IA y de las ciencias de la computación centrada en el uso de datos y algoritmos para imitar la forma en la que los humanos aprendemos, detectando patrones o regularidades para realizar predicciones.

Existen 3 tipos de aprendizaje dentro del ML \cite{18,19}:

El \textbf{aprendizaje supervisado} consiste en entrenar con datos de los que sabemos sus etiquetas (una etiqueta nos indica qué es cada dato). Por ejemplo, los datos de entrada podrían ser imágenes de animales y sus etiquetas podrían ser "perro" o "gato". A partir de los datos y sus etiquetas, el agente aprende una función que dado un nuevo dato, predice su etiqueta. Es el tipo de aprendizaje más utilizado, los datos vienen ya 'preparados' para su uso.

En el \textbf{aprendizaje no supervisado} el agente aprende los patrones de los datos de entrada sin ninguna realimentación, es decir, los datos no están etiquetados. La herramienta más usada en el aprendizaje no supervisado es el agrupamiento, que consiste en detectar potenciales grupos en los datos de entrada. Este enfoque requiere un mayor número de datos

En el \textbf{aprendizaje por refuerzo} el agente aprende mediante una serie de recompensas o castigos. El agente intentará realizar acciones que le proporcionen mejores recompensas en el futuro. Este tipo de aprendizaje es muy utilizado para enseñar a jugar a juegos.


\begin{comment}
\section{Visión por computador}
La visión por computador (VC) \cite{15} es un campo de la inteligencia artificial (IA) que permite a las computadoras y sistemas obtener información significativa a partir de imágenes digitales, vídeos y otras entradas visuales, y realizar acciones en base a esa información.

La visión por computador requiere una gran cantidad de datos. Realiza el análisis de los datos una y otra vez hasta que distingue diferencias o patrones y, en última instancia, reconoce imágenes. 

Se utilizan dos tecnologías esenciales para lograr esto: un tipo de aprendizaje automático llamado aprendizaje profundo (deep learning, DL) y las redes neuronales convolucionales (convolutional neuronal networks, CNNs).
\end{comment}


\section{Aprendizaje profundo}

\subsection{Redes neuronales}
Las redes neuronales son redes computacionales que intentan, a groso modo, simular el proceso de decisión de las neuronas del sistema nervioso central de los animales o humanos. Ventajas de las redes neuronales: son computacionalmente y algoritmicamente muy simples.

Las neuronas son la unidad fundamental de cómputo,y están conectadas juntas en la red para procesar datos.


Foto de neuronas conectadas
Foto de red neuronal


\subsection{Redes neuronales convolucionales}


\subsubsection*{Capa de pooling}