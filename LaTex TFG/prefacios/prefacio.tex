\chapter*{}
\thispagestyle{empty}


%\thispagestyle{empty}

\begin{center}
{\large\bfseries Estimación de distancia cámara-sujeto en fotografías faciales usando aprendizaje profundo}\\
\end{center}
\begin{center}
Iván Salinas López\\
\end{center}

%\vspace{0.7cm}
\noindent{\textbf{Palabras clave}: aprendizaje automático, aprendizaje profundo, distorsión de perspectiva, estimación de distancia cámara-sujeto, regresión, visión por computador.}\\

\vspace{0.7cm}
\noindent{\textbf{Resumen}}\\

La relevancia de las imágenes faciales ha aumentado significativamente a lo largo de los años, gracias a su amplio uso en diversos contextos. En particular, la estimación de la distancia entre la cámara y el sujeto es un factor crucial en estas fotografías. Esta predicción permite calcular la distorsión de los sujetos en las imágenes y, en consecuencia, desarrollar avances tanto en la corrección de dichas distorsiones como en ámbitos de identificación o reconocimiento facial.

En este TFG se ha generado un conjunto de datos de imágenes completamente sintético, diverso y realista, a partir de diferentes modelos 3D tanto faciales como de cuerpo completo. Estas imágenes se generaron a distancias que van desde 50 cm hasta 600 cm. A partir de este conjunto, se desarrolló el método FacialSCDnet+, basado en FacialSCDnet, que utiliza dos modelos de aprendizaje profundo para estimar automáticamente la distancia. En concreto, se emplearon las arquitecturas VGG-16 y ResNet-50 adaptadas para el problema de regresión. El método FacialSCDnet+ superó el rendimiento del método anterior, tanto en predicciones sobre conjuntos de imágenes sintéticas como en conjuntos de imágenes reales. En concreto, en este último conjunto se logró una distorsión de 0.007 frente a 0.012 del conjunto FacialSCDnet, y un error medio absoluto de 26.05 cm frente a 31.504 cm.

En definitiva, la estimación de la distancia cámara-sujeto en fotografías faciales aún tiene un amplio margen de mejora, especialmente en el campo del aprendizaje profundo. Por tanto, se trata de una prometedora línea de investigación.
\cleardoublepage


\thispagestyle{empty}


\begin{center}
{\large\bfseries Camera-subject distance estimation in facial photographs using deep learning}\\
\end{center}
\begin{center}
Iván Salinas López\\
\end{center}

%\vspace{0.7cm}
\noindent{\textbf{Keywords}: machine learning, deep learning, perspective distortion, camera-subject distance estimation, regression, computer vision.}\\

\vspace{0.7cm}
\noindent{\textbf{Abstract}}\\

The significance of facial images has increased significantly over the years, thanks to their widespread use in various contexts. In particular, estimating the distance between the camera and the subject is a crucial factor in these photographs. This prediction allows for calculating the distortion of subjects in the images and, consequently, developing advances both in correcting such distortions and in fields like identification or facial recognition.

In this Final Degree Project, a completely synthetic, diverse, and realistic dataset of images was generated from various 3D models, including both facial and full-body models. These images were generated at distances ranging from 50 cm to 600 cm. From this dataset, the method FacialSCDnet+ was developed, based on FacialSCDnet, which uses two deep learning models to automatically estimate the distance. Specifically, the VGG-16 and ResNet-50 architectures adapted for the regression problem were used. The FacialSCDnet+ method outperformed the previous method, both in predictions on synthetic image sets and on real image sets. Specifically, in the latter set, a distortion of 0.007 was achieved compared to 0.012 of the FacialSCDnet set, and a mean absolute error of 26.05 cm compared to 31.504 cm.

In conclusion, estimating the camera-subject distance in facial photographs still has ample room for improvement, especially in the field of deep learning. Therefore, it is a promising line of research.

\chapter*{}
\thispagestyle{empty}

\noindent\rule[-1ex]{\textwidth}{2pt}\\[4.5ex]

Yo, \textbf{Iván Salinas López}, alumno de la titulación Grado en Ingeniería Informática de la \textbf{Escuela Técnica Superior de Ingenierías Informática y de Telecomunicación de la Universidad de Granada}, con DNI 78026145W, autorizo la ubicación de la siguiente copia de mi Trabajo Fin de Grado en la biblioteca del centro para que pueda ser consultada por las personas que lo deseen.

\vspace{6cm}

\noindent Fdo: Iván Salinas López

\vspace{2cm}

\begin{flushright}
Granada a X de Junio de 2024
\end{flushright}


\chapter*{}
\thispagestyle{empty}

\noindent\rule[-1ex]{\textwidth}{2pt}\\[4.5ex]

D. \textbf{Enrique Bermejo Nievas}, Investigador Senior en Panacea Cooperative Research y miembro del Instituto Andaluz Interuniversitario en Ciencia de Datos e Inteligencia Computacional.

\vspace{0.5cm}

D. \textbf{Pablo Mesejo Santiago}, Profesor del Área de Ciencias de la Computación e Inteligencia Artificial del Departamento Ciencias de la Computación e Inteligencia Artificial de la Universidad de Granada.


\vspace{0.5cm}

\textbf{Informan:}

\vspace{0.5cm}

Que el presente trabajo, titulado \textit{\textbf{Estimación de la distancia cámara-sujeto en fotografías faciales mediante técnicas de aprendizaje profundo}}, ha sido realizado bajo su supervisión por \textbf{Iván Salinas López}, y autorizamos la defensa de dicho trabajo ante el tribunal
que corresponda.

\vspace{0.5cm}

Y para que conste, expiden y firman el presente informe en Granada a X de Junio de 2024.

\vspace{1cm}

\textbf{Los directores:}

\vspace{5cm}

\noindent \textbf{Enrique Bermejo Nievas \ \ \ \ \ Pablo Mesejo Santiago}

\chapter*{Agradecimientos}
\thispagestyle{empty}

       \vspace{1cm}


Poner aquí agradecimientos...

